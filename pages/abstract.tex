\abstract{Knowledge of the structure of nucleons ($i.e.$ protons and neutrons) is a central topic of interest to nuclear/particle physicists. Much more is known about the structure of the proton than the neutron due to the lack of high-density free neutron targets. The Barely Off-shell Nucleon Structure experiment (BONuS12) at Jefferson Lab (JLab) is a second generation experiment upgraded/optimized to advance our knowledge of the neutron's structure using the deep-inelastic scattering of electrons off deuterium. Typically, since deuterium is a nuclear target, corrections for off-shell and nuclear binding effects must be taken into account in order to extract results on the neutron. These corrections are model-dependent and therefore have limited our success in extracting neutron information using deuterium targets.\\
\indent In the BONuS12 experiment, 10.6 GeV electrons are scattered off of a deuterium target. By detecting the low momentum spectator proton at backward angles, the uncertainty due to final state interactions is minimized. The goal of the experiment is to measure the ratio of the neutron to proton structure functions ($F_2^n/F_2^p$) as the Bjroken scaling variable $x$ approaches $1$. The newly designed Radial Time Projection Chamber (RTPC) for BONuS12 detects the spectator proton in coincidence with the scattered electron, which is detected in the CEBAF Large Acceptance Spectrometer (CLAS12).\\
\indent This work presents the simulation and development of the new BONuS12 RTPC. The design, construction, and testing of the Drift-gas Monitoring Sysytem (DMS) for the BONuS12 experiment is also described. The results of the DMS operation as well as the first preliminary data from the BONuS12 experimental run are given. Because the BONuS12 data analysis depends on CLAS12 working effectively, an effort to verify the CLAS12 operation with the extraction of the inclusive deep inelastic cross section from the first experiment in CLAS12 (Run Group A) will be presented.}
